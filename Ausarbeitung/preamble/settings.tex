%%%%%%%%%%%%%%%%%%%%%%%%%%%%%%%%%%%%%%%%%%%
%              Variablen                  %
%%%%%%%%%%%%%%%%%%%%%%%%%%%%%%%%%%%%%%%%%%%

%Autor
\renewcommand{\author}{Max Mustermann} %Name
\newcommand{\city}{Stadt} %Stadt
\newcommand{\dateOfBirth}{Geburtsdatum} %Geburtsdatum
\newcommand{\matrNumber}{Matrikelnummer} %Matrikelnummer
\newcommand{\studycourse}{Studiengang} %Studiengang

%Thema
\renewcommand{\subject}{Bachelor Thesis} %Art/Thema der Arbeit
\newcommand{\titel}{Titel der Arbeit} %Titel der Arbeit
\renewcommand{\subtitle}{Untertitel} %Untertitel
\newcommand{\degree}{Bachelor of Science\par{}B. Sc.} %Angestrebter Titel (nur bei Abschlussarbeiten, sonst leer lassen/auskommentieren)

%Firmeninformationen (ggf. für Sperrvermerk)
\newcommand{\company}{Firmenname für Sperrvermerk} %Firmenname

%Betreuer
\newcommand{\supervisor}{Betreuer} %Betreuer
\newcommand{\institution}{Uni} %Hochschule
\newcommand{\faculty}{Fachbereich} %Fachbereich
\newcommand{\toponym}{Ort} %Ort

%Zweitprüfer
\newcommand{\secondauditor}{Zweitprüfer} %Prüfer

%Überschriften
\renewcommand{\lstlistlistingname}{Quellcodeverzeichnis}
\renewcommand{\lstlistingname}{Quellcode}
\newcommand{\listpseudocodename}{Pseudocodeverzeichnis}
\renewcommand{\listtablename}{Tabellenverzeichnis}
\renewcommand{\listfigurename}{Abbildungsverzeichnis}
\newcommand{\lstlistglossaryname}{Glossar}
\newcommand{\lstlistglossaryheading}{Glossarverzeichnis}
\newcommand{\lstlistacronymname}{Abkürzungsverzeichnis}
\newcommand{\lstlistsymbolname}{Symbolverzeichnis}
\renewcommand{\listalgorithmcfname}{Algorithmenverzeichnis}
\DefineBibliographyStrings{ngerman}{bibliography = {Literaturverzeichnis}} %ggf. references = {Literaturverzeichnis}

% Algorithm2e
\SetAlCapSkip{1em}
\SetAlCapSty{}
\setlength\algomargin{2em}

% Sonsitges
\newcommand{\chapterorig}{\chapter}

%%%%%%%%%%%%%%%%%%%%%%%%%%%%%%%%%%%%%%%%%%%
%               Einstellungen             %
%%%%%%%%%%%%%%%%%%%%%%%%%%%%%%%%%%%%%%%%%%%

% **************************** HYPERREF SETUP *******************************
\definecolor{linkcolor}{RGB}{0,0,0}
\definecolor{urlcolor}{RGB}{0,118,50}
\definecolor{filecolor}{RGB}{174,15,10}
\hypersetup
{
%bookmarks=true,                       % Lesezeichen im PDF erzeugen
bookmarksopen=true,                    % Lesezeichen im PDF sofort anzeigen
%backref=true,                         % Rückverweise im Literaturverzeichnis
colorlinks=true,                       % Farbige Verweise
%hidelinks = true,                     % Verweise verbergen (entfernt Farbe und Rahmen)
pdfstartview={FitH},                   % Ansicht des PDFs beim öffnen
linkcolor=linkcolor,                   % Farbe von Querverweisen
citecolor=linkcolor,                   % Farbe von Zitaten
filecolor=filecolor,                   % Farbe von Verweisen auf Dateien
urlcolor=urlcolor                      % Farbe von URLs
}

\filecontentsspecials|[]
\begin{filecontents*}[overwrite]{\jobname.xmpdata}
  \Title{|titel}
  \Author{|author}
  \Subject{|subject}
  \Keywords{key\sep words\sep etc} 
  \Language{de-DE}
  \Copyrighted{True}
\end{filecontents*}

% **************************** GRAPHICX SETUP *******************************
\DeclareGraphicsExtensions{.pdf,.png,.jpg,.jpeg} % bekannte Graphik-Dateiformate (müssen nicht mehr im Dateinamen angegeben werden, also statt "beispiel.png" nur noch "beispiel")
\graphicspath{{./figure/}}   % path to graphics folder, usage {PATH},{ANOTHERPATH}...
%\captionsetup{format=plain} % Captions werden nicht eingerückt

% *************************** MATH SETUP ***************************************
\DeclarePairedDelimiter{\norm}{\lVert}{\rVert}
\pgfplotsset{compat=newest}
\pgfkeys{/pgf/number format/.cd,use comma, 1000 sep={.}} % german number formats
\pgfplotsset{colormap/bluered} % default colormap

% **************************** LISTINGS SETUP *******************************
\lstset{literate=
  {á}{{\'a}}1 {é}{{\'e}}1 {í}{{\'i}}1 {ó}{{\'o}}1 {ú}{{\'u}}1
  {Á}{{\'A}}1 {É}{{\'E}}1 {Í}{{\'I}}1 {Ó}{{\'O}}1 {Ú}{{\'U}}1
  {à}{{\`a}}1 {è}{{\`e}}1 {ì}{{\`i}}1 {ò}{{\`o}}1 {ù}{{\`u}}1
  {À}{{\`A}}1 {È}{{\'E}}1 {Ì}{{\`I}}1 {Ò}{{\`O}}1 {Ù}{{\`U}}1
  {ä}{{\"a}}1 {ë}{{\"e}}1 {ï}{{\"i}}1 {ö}{{\"o}}1 {ü}{{\"u}}1
  {Ä}{{\"A}}1 {Ë}{{\"E}}1 {Ï}{{\"I}}1 {Ö}{{\"O}}1 {Ü}{{\"U}}1
  {â}{{\^a}}1 {ê}{{\^e}}1 {î}{{\^i}}1 {ô}{{\^o}}1 {û}{{\^u}}1
  {Â}{{\^A}}1 {Ê}{{\^E}}1 {Î}{{\^I}}1 {Ô}{{\^O}}1 {Û}{{\^U}}1
  {œ}{{\oe}}1 {Œ}{{\OE}}1 {æ}{{\ae}}1 {Æ}{{\AE}}1 {ß}{{\ss}}1
  {ű}{{\H{u}}}1 {Ű}{{\H{U}}}1 {ő}{{\H{o}}}1 {Ő}{{\H{O}}}1
  {ç}{{\c c}}1 {Ç}{{\c C}}1 {ø}{{\o}}1 {å}{{\r a}}1 {Å}{{\r A}}1
  {€}{{\euro}}1 {£}{{\pounds}}1 {«}{{\guillemotleft}}1
  {»}{{\guillemotright}}1 {ñ}{{\~n}}1 {Ñ}{{\~N}}1 {¿}{{?`}}1
}

\definecolor{listingbgcolor}{RGB}{242,242,242}
\definecolor{black}{RGB}{0,0,0}
\definecolor{strings}{RGB}{0,0,1}
\definecolor{comments}{rgb}{0.25,0.5,0.37}
\definecolor{keywords}{rgb}{0.5,0.0,0.3}

\lstset{ %
    backgroundcolor=\color{listingbgcolor},   % Hintergrundfarbe
    basicstyle=\linespread{0.94}\footnotesize\ttfamily, % Schrifteinstellungen für Quellcode
    breakatwhitespace=false,         % Automatische Zeilenumbrüche nur bei Leer- oder Tabulatorzeichen (Leerraum/whitespaces)
    breaklines=true,                 % Automatische Zeilenumbrüche
    captionpos=b,                    % Beschriftung unten
    commentstyle=\color{comments},   % Schrifteinstellungen für Kommentare
    columns=flexible,                 % Ist notwendig, damit man Quellcode aus den Listings kopieren kann
    %  deletekeywords={...},            % Bestimmte Schlüsselwörter entfernen
    escapeinside={\%*}{*)},          % Defintion von Escape-Sequenzen
    extendedchars=true,                   % Nicht ASCII-Zeichen erlauben
    frame=single,                    % Rahmen um den Quellcode
    keepspaces=true,                 % Einrückungen im Quellcode behalten
    keywordstyle=\bfseries\color{keywords},% Schrifteinstellungen für Schlüsselwörter
    language=java,                   % Programmiersprache des Quellcodes
    %  morekeywords={*,...},            % Zusätzliche Schlüsselwörter
    numbers=left,                    % Zeilennummerierung
    numbersep=5pt,                   % Abstand zwischen Zeilennummerierung und Quellcode
    numberstyle=\color{black}, % Schrifteinstellungen für Zeilennummern
    rulecolor=\color{black},         % if not set, the frame-color may be changed on line-breaks within not-black text (e.g. comments (green here))
    showspaces=false,                % Leerraum-Zeichen anzeigen
    showstringspaces=false,          % Leerzeichen in Zeichenketten anzeigen
    showtabs=false,                  % Tabulatorzeichen in Zeichenketten anzeigen
    stepnumber=1,                    % Schrittweite bei Zeilennummern
    stringstyle=\color{strings},        % Schrifteinstellungen für Zeichenketten
    tabsize=4,                       % Tabulatorbreite (Anzahl Leerzeichen)
    numberbychapter=false            % Nummeriere Quellcode fortlaufend je Kapitel
}

\definecolor{comments_java}{RGB}{5,132,71}
\definecolor{keywords_java}{RGB}{112, 6, 147}
\definecolor{strings_java}{RGB}{0, 0, 0}
\definecolor{identifiers_java}{rgb}{0, 0, 1}
\lstdefinestyle{java}
{
    language=Java,
    keywordstyle=\bfseries\color{keywords_java},  	% underlined bold black keywords 
    identifierstyle=\bfseries\color{identifiers_java}, 
    commentstyle=\bfseries\color{comments_java}, % white comments 
    stringstyle=\bfseries\color{strings_java},
}

\definecolor{basic_xml}{RGB}{5,132,71}
\definecolor{keywords_xml}{RGB}{112, 6, 147}
\definecolor{identifiers_xml}{rgb}{0, 0, 1}
\definecolor{strings_xml}{rgb}{0, 0, 1}
\definecolor{comments_xml}{RGB} {63,95,191}
\definecolor{rulecolor_xml}{RGB} {0,0,0}
\lstdefinestyle{xml}
{
    language=xml,
    basicstyle=\fontsize{9pt}{9pt}\selectfont\color{basic_xml},
    keywordstyle=\color{keywords_xml},  	% underlined bold black keywords 
    %Hier können bei Bedarf noch weitere Keywords eingetragen werden
    keywords={name, value, version, encoding, id, type, xmlns:xsi, ref, namespace},
    identifierstyle=\color{identifiers_xml},  
    stringstyle=\color{strings_xml},  
    commentstyle=\color{comments_xml},
    morecomment=[s]{<!--}{-->},
    rulecolor=\color{rulecolor_xml}
}

\AtBeginDocument{\numberwithin{lstlisting}{chapter}} % Nummeriere Quellcode fortlaufend je Abschnitt

% *************************** PSEUDOCODE SETUP ********************************
\floatstyle{boxed}                        % Rahmen für pseudocode-Umgebung
\newfloat{pseudocode}{htbp}{lop}[section] % Definieren pseudocode-Umgebung
\floatname{pseudocode}{Pseudocode}        % Beschrifte pseudocode-Umgebung mit "Pseudocode"

\newcommand{\listofpseudocodename}{Pseudocodeverzeichnis}
\newcommand{\listofpseudocode}{\listof{pseudocode}{\listofpseudocodename}}
\providecommand*{\pseudocodeautorefname}{Pseudocode}

% ************************* BIBLIOGRAPHY SETUP ******************************
\addbibresource{bibliography.bib}
%\nocite{*} % Vor der Abgabe entfernen - zeigt auch nichtbenutzte Quellen an
\setcounter{biburllcpenalty}{7000}
\setcounter{biburlucpenalty}{8000}

% ***************************** PAGE SETUP **********************************
\setlength\parindent{0pt} % Abstand der Überschriften zum Header
%\renewcommand{\baselinestretch}{1.5} %Setzt den Zeilenabstand global 
\widowpenalty=300
\clubpenalty=300

\geometry{
	a4paper,
	paperwidth=17cm, % Für book, auskommentieren für scrreprt
	paperheight=24cm, % Für book, auskommentieren für scrreprt
	textwidth=12cm, % 12cm für book, 14cm für scrreprt
	textheight=19cm, % 19cm für book, 23cm für scrreprt
%	left=3cm,
	right=2cm, % 2cm für book, 3cm für scrreprt
	top=2cm, % 2cm für book
%	bottom=3cm,
%	margin=2cm
}


\pagestyle{scrheadings}
\clearpairofpagestyles% Leeren von Kopf- und Fußzeile
\renewcommand{\chapterheadstartvskip}{}
\ihead{\headmark}
\ohead{\pagemark}

\newpairofpagestyles{pagenumbers}{
    \ohead{\pagemark}
}
\renewcommand*{\chapterpagestyle}{pagenumbers}

\setlength{\headsep}{15mm}
\addtokomafont{pagenumber}{\normalfont\bfseries\sffamily} % Anpassen der Schriftart der Seitenzahl
\setkomafont{pagehead}{\normalfont\bfseries\sffamily}  % Setzen der Schriftart für Kopfzeile
\automark[section]{chapter}

% Schriftarten
\usepackage{mathptmx} % Hier steckt Times drin
\usepackage[scaled]{helvet}
\usepackage{courier}

% ***************************** ITEMIZE SETUP********************************
\newcommand{\itemizesquare}{\textcolor{lightgray}{\raisebox{.45ex}{\rule{1ex}{1ex}}}}
\renewcommand{\labelitemi}{\itemizesquare}
