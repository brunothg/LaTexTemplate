% Zum bauen des Glossars müssen folgende Kompilierschritte durchlaufen werden:
% 1. pdflatex file
% 2. makeglossaries file
% 3. pdflatex file
% 4. pdflatex file

%\GlsSetQuote{+} %for ngerman " override
\newglossary[slg]{symbolslist}{syi}{syg}{Symbolverzeichnis}
\makeglossaries
\makeindex

%%%%%%%%%%%%%%%%%%%%%%%%%%%%%%%%%%%%%%%%%%%
%---------------Symbole-------------------%
%%%%%%%%%%%%%%%%%%%%%%%%%%%%%%%%%%%%%%%%%%%
%\newglossaryentry{symb:Pi}{
%    name=$\pi$,
%    description={Die Kreiszahl.},
%    sort=symbolpi, type=symbolslist
%}



%%%%%%%%%%%%%%%%%%%%%%%%%%%%%%%%%%%%%%%%%%%
%---------------Glossar-------------------%
%%%%%%%%%%%%%%%%%%%%%%%%%%%%%%%%%%%%%%%%%%%
%\newglossaryentry{glos:AntwD}{
%     name=Antwortdatei
%    ,description={Informationen zum
%        Installieren einer Anwendung oder des Betriebssystems.}
%}
\newglossaryentry{glos:CFT}{
    name={Cross-funktionales Team},
    plural={Cross-funktionalen Teams},
    description={Ein Cross-funktionales Team ist eine Gruppe von Mitarbeitern aus verschiedenen Funktionsbereichen des Unternehmens (z.B. Finanzen, Personal, Marketing, Entwicklung und Vertrieb). Sie sind meistens selbstbestimmt und dafür verantwortlich, ihre Ziele umzusetzen.}
}

%%%%%%%%%%%%%%%%%%%%%%%%%%%%%%%%%%%%%%%%%%%
%---------------Abkürzungen---------------%
%%%%%%%%%%%%%%%%%%%%%%%%%%%%%%%%%%%%%%%%%%%
\newacronym[]{CFT}{CFT}{Cross-funktionales Team\protect\glsadd{glos:CFT}}
