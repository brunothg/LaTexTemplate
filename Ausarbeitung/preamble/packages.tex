\usepackage{morewrites} % Notwendig für "zu viele" Schreibintensive packages

% Unterstützung für die deutsche Sprache
\usepackage[ngerman]{babel} % Lokalisierung von Typographie, Silbentrennung, ...
\usepackage[utf8]{inputenc} % Unterstützung von UTF-8 in Eingabe-Dateien
\usepackage[T1]{fontenc} % Zeichensatzkodierung von LaTeX (Cork-Kodierung)

% Schriftarten
\usepackage{lmodern}
\usepackage{mathptmx}
\usepackage[scaled]{helvet}
\usepackage{courier}

% Bibliography
\usepackage[autostyle]{csquotes}
\usepackage[
	style=alphabetic
	,sorting=nty
	,sortlocale=de_DE
	,natbib=true
	,backend=biber
	,maxbibnames=99
	,block=ragged % Kein Blocktext im Quellenverzeichnis
]{biblatex} % Moderne bibtex version

% Formatierung
\usepackage[pdfa]{hyperref} % Querverweise, Hyperlink, pdf-Konfiguration, etc.
\usepackage[
xindy         %Unicode index sorter
,nonumberlist %keine Seitenzahlen anzeigen
,nopostdot    %Den Punkt am Ende jeder Beschreibung deaktivieren
,acronym      %ein Abkürzungsverzeichnis erstellen
,toc          %Einträge im Inhaltsverzeichnis
,numberedsection % Nummerierung
,section      %im Inhaltsverzeichnis auf section-Ebene erscheinen
%,automake     %erstellt wenn möglich alle Verzeichnisse ohne expliziten makeglossaries Aufruf
%,numberedsection=autolabel %Zum einfügen in das Appendix
]{glossaries} % vor hyperref einbinden, um Verlinkung zu deaktiveren
\usepackage{geometry} % Zum Gestalten der Seiten
\usepackage{setspace} %  Zeilenabstand zu ändern
\usepackage{ellipsis}	% Korrigiert den Weißraum um Auslassungspunkte
\usepackage{scrhack} % Bugfix für listings unter KomaScript\textsl{•}
\usepackage{listings} % Quellcode
\usepackage{float} % Floats are containers for things in a document that cannot be broken over a page
\usepackage[headsepline]{scrlayer-scrpage} % KomaScript Kopf- & Fußzeile
\usepackage{booktabs} % Professionelle Tabellen; http://www.tablesgenerator.com/latex_tables
\usepackage{soulutf8}
\usepackage[section]{placeins} % verhindert, dass floats über Abschnittsgrenzen hinweg verschoben werden; \FloatBarrier für eigene Grenzen

% Programmierung, Pseudocode, etc ...
\usepackage[
ngerman,
germankw,
onelanguage, % Ausschalten, wenn englische keywords gewünscht sind
algo2e,
linesnumbered,
longend,
boxed,
vlined,
algochapter
]{algorithm2e}

% Grafisches
\usepackage{caption}
\usepackage{graphicx} % Erweiterte Unterstützung von Graphiken
\usepackage{rotating} % Drehen von Graphiken inkl. Unterschrift
\usepackage[table, xcdraw]{xcolor} % TeX-Engine-unabhängige Definition von Farben
% \usepackage{svg} % Für Vektorgrafiken
% \usepackage{epstopdf} % Für EPS Dateien

% Mathematik, Formeln, etc.
\usepackage{amsmath, amsfonts, amssymb, amsthm} % Paket für die align Umgebung
\usepackage{mathtools}
\usepackage{pgfplots}

% Sonstige
\usepackage{blindtext} % Erzeugt Platzhalter Text
\usepackage{pgffor} % foreach schleife
\usepackage[a-2u]{pdfx} % Erzeugt PDF/A-1b

%\usepackage{filecontents}
\def\filecontentsspecials#1#2#3{
  \global\let\ltxspecials\dospecials
  \gdef\dospecials{\ltxspecials
    \catcode`#1=0
    \catcode`#2=1
    \catcode`#3=2
    \global\let\dospecials\ltxspecials
  }
}
